\documentclass[]{article}
\usepackage{lmodern}
\usepackage{amssymb,amsmath}
\usepackage{ifxetex,ifluatex}
\usepackage{fixltx2e} % provides \textsubscript
\ifnum 0\ifxetex 1\fi\ifluatex 1\fi=0 % if pdftex
  \usepackage[T1]{fontenc}
  \usepackage[utf8]{inputenc}
\else % if luatex or xelatex
  \ifxetex
    \usepackage{mathspec}
  \else
    \usepackage{fontspec}
  \fi
  \defaultfontfeatures{Ligatures=TeX,Scale=MatchLowercase}
\fi
% use upquote if available, for straight quotes in verbatim environments
\IfFileExists{upquote.sty}{\usepackage{upquote}}{}
% use microtype if available
\IfFileExists{microtype.sty}{%
\usepackage{microtype}
\UseMicrotypeSet[protrusion]{basicmath} % disable protrusion for tt fonts
}{}
\usepackage[margin=1in]{geometry}
\usepackage{hyperref}
\hypersetup{unicode=true,
            pdftitle={The evolution of badges of status with learners},
            pdfauthor={Andres Quiñones},
            pdfborder={0 0 0},
            breaklinks=true}
\urlstyle{same}  % don't use monospace font for urls
\usepackage{graphicx,grffile}
\makeatletter
\def\maxwidth{\ifdim\Gin@nat@width>\linewidth\linewidth\else\Gin@nat@width\fi}
\def\maxheight{\ifdim\Gin@nat@height>\textheight\textheight\else\Gin@nat@height\fi}
\makeatother
% Scale images if necessary, so that they will not overflow the page
% margins by default, and it is still possible to overwrite the defaults
% using explicit options in \includegraphics[width, height, ...]{}
\setkeys{Gin}{width=\maxwidth,height=\maxheight,keepaspectratio}
\IfFileExists{parskip.sty}{%
\usepackage{parskip}
}{% else
\setlength{\parindent}{0pt}
\setlength{\parskip}{6pt plus 2pt minus 1pt}
}
\setlength{\emergencystretch}{3em}  % prevent overfull lines
\providecommand{\tightlist}{%
  \setlength{\itemsep}{0pt}\setlength{\parskip}{0pt}}
\setcounter{secnumdepth}{0}
% Redefines (sub)paragraphs to behave more like sections
\ifx\paragraph\undefined\else
\let\oldparagraph\paragraph
\renewcommand{\paragraph}[1]{\oldparagraph{#1}\mbox{}}
\fi
\ifx\subparagraph\undefined\else
\let\oldsubparagraph\subparagraph
\renewcommand{\subparagraph}[1]{\oldsubparagraph{#1}\mbox{}}
\fi

%%% Use protect on footnotes to avoid problems with footnotes in titles
\let\rmarkdownfootnote\footnote%
\def\footnote{\protect\rmarkdownfootnote}

%%% Change title format to be more compact
\usepackage{titling}

% Create subtitle command for use in maketitle
\providecommand{\subtitle}[1]{
  \posttitle{
    \begin{center}\large#1\end{center}
    }
}

\setlength{\droptitle}{-2em}

  \title{The evolution of badges of status with learners}
    \pretitle{\vspace{\droptitle}\centering\huge}
  \posttitle{\par}
    \author{Andres Quiñones}
    \preauthor{\centering\large\emph}
  \postauthor{\par}
    \date{}
    \predate{}\postdate{}
  
\usepackage{float}

\floatplacement{figure}{H}

\begin{document}
\maketitle

\hypertarget{the-hawk-dove-game}{%
\section{The Hawk-Dove game}\label{the-hawk-dove-game}}

Classical Hawk-Dove game. Hawks compete with doves in a population of
reproductive individuals.

\begin{figure}
\centering
\includegraphics{Simulations//mutType_//basicHawkDove.png}
\caption{Hawk-dove game}
\end{figure}

\hypertarget{the-effect-of-learning}{%
\section{The effect of learning}\label{the-effect-of-learning}}

A population of individuals that estimate a value and a preference for a
particular behaviour. In our case they specify the probability of
playing dove in the Hawk-Dove game.

\hypertarget{result-when-individuals-do-not-vary-in-their-quality}{%
\subsection{Result when individuals do NOT vary in their
quality}\label{result-when-individuals-do-not-vary-in-their-quality}}

\begin{figure}
\centering
\includegraphics{Simulations//QualStDv_//hawkDoveLearn.png}
\caption{Learning equilibrium when individuals do not vary in quality}
\end{figure}

\hypertarget{results-when-individuals-vary-in-their-quality}{%
\subsection{Results when individuals vary in their
quality}\label{results-when-individuals-vary-in-their-quality}}

\begin{figure}
\centering
\includegraphics{Simulations//QualStDv_//hawkDoveLearn_0.1.png}
\caption{Learning equilibrium when individuals vary in quality}
\end{figure}

\hypertarget{over-all-effect-of-variance}{%
\subsection{Over all effect of
variance}\label{over-all-effect-of-variance}}

\begin{figure}
\centering
\includegraphics{Simulations//QualStDv_//effectQualVariance.png}
\caption{Effect of variation in quality for the learning equilibrium}
\end{figure}

\hypertarget{how-does-learning-work}{%
\subsection{How does learning work?}\label{how-does-learning-work}}

\begin{figure}
\centering
\includegraphics{cartoonRBF.png}
\caption{Function approximation for0 the actor and the critic}
\end{figure}

\hypertarget{how-does-learning-look-like-in-the-hawk-dove-game}{%
\subsection{How does learning look like in the Hawk-Dove
game}\label{how-does-learning-look-like-in-the-hawk-dove-game}}

\begin{figure}
\centering
\includegraphics{Simulations//QualStDv_//WeightsVarQualSt.png}
\caption{Within population variation in the learning parameters at the
end of learning for three scenarios of variation in the quality of
individuals}
\end{figure}

\hypertarget{how-do-learners-fare-against-the-pure-types}{%
\section{How do learners fare against the ``pure''
types}\label{how-do-learners-fare-against-the-pure-types}}

\begin{figure}
\centering
\includegraphics{Simulations//mutType_//evolDynTypes.png}
\caption{Evolutionary dynamics of the learners competing against pure
strategies}
\end{figure}

\hypertarget{how-do-the-learning-dynamics-look-like}{%
\subsection{How do the learning dynamics look
like}\label{how-do-the-learning-dynamics-look-like}}

\begin{figure}
\centering
\includegraphics{Simulations//mutType_//learnDyn200.png}
\caption{Learning dynamics of the individuals in a population}
\end{figure}

\hypertarget{whats-next}{%
\section{What's next?}\label{whats-next}}

Let reaction norm evolve, under different initial conditions.


\end{document}
