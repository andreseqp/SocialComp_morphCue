% Options for packages loaded elsewhere
\PassOptionsToPackage{unicode}{hyperref}
\PassOptionsToPackage{hyphens}{url}
%
\documentclass[
  ignorenonframetext,
]{beamer}
\usepackage{pgfpages}
\setbeamertemplate{caption}[numbered]
\setbeamertemplate{caption label separator}{: }
\setbeamercolor{caption name}{fg=normal text.fg}
\beamertemplatenavigationsymbolsempty
% Prevent slide breaks in the middle of a paragraph
\widowpenalties 1 10000
\raggedbottom
\setbeamertemplate{part page}{
  \centering
  \begin{beamercolorbox}[sep=16pt,center]{part title}
    \usebeamerfont{part title}\insertpart\par
  \end{beamercolorbox}
}
\setbeamertemplate{section page}{
  \centering
  \begin{beamercolorbox}[sep=12pt,center]{part title}
    \usebeamerfont{section title}\insertsection\par
  \end{beamercolorbox}
}
\setbeamertemplate{subsection page}{
  \centering
  \begin{beamercolorbox}[sep=8pt,center]{part title}
    \usebeamerfont{subsection title}\insertsubsection\par
  \end{beamercolorbox}
}
\AtBeginPart{
  \frame{\partpage}
}
\AtBeginSection{
  \ifbibliography
  \else
    \frame{\sectionpage}
  \fi
}
\AtBeginSubsection{
  \frame{\subsectionpage}
}
\usepackage{lmodern}
\usepackage{amssymb,amsmath}
\usepackage{ifxetex,ifluatex}
\ifnum 0\ifxetex 1\fi\ifluatex 1\fi=0 % if pdftex
  \usepackage[T1]{fontenc}
  \usepackage[utf8]{inputenc}
  \usepackage{textcomp} % provide euro and other symbols
\else % if luatex or xetex
  \usepackage{unicode-math}
  \defaultfontfeatures{Scale=MatchLowercase}
  \defaultfontfeatures[\rmfamily]{Ligatures=TeX,Scale=1}
\fi
% Use upquote if available, for straight quotes in verbatim environments
\IfFileExists{upquote.sty}{\usepackage{upquote}}{}
\IfFileExists{microtype.sty}{% use microtype if available
  \usepackage[]{microtype}
  \UseMicrotypeSet[protrusion]{basicmath} % disable protrusion for tt fonts
}{}
\makeatletter
\@ifundefined{KOMAClassName}{% if non-KOMA class
  \IfFileExists{parskip.sty}{%
    \usepackage{parskip}
  }{% else
    \setlength{\parindent}{0pt}
    \setlength{\parskip}{6pt plus 2pt minus 1pt}}
}{% if KOMA class
  \KOMAoptions{parskip=half}}
\makeatother
\usepackage{xcolor}
\IfFileExists{xurl.sty}{\usepackage{xurl}}{} % add URL line breaks if available
\IfFileExists{bookmark.sty}{\usepackage{bookmark}}{\usepackage{hyperref}}
\hypersetup{
  pdftitle={Badges of status and Individual Recognition mediated by learning},
  pdfauthor={Andrés Quiñones},
  hidelinks,
  pdfcreator={LaTeX via pandoc}}
\urlstyle{same} % disable monospaced font for URLs
\newif\ifbibliography
\setlength{\emergencystretch}{3em} % prevent overfull lines
\providecommand{\tightlist}{%
  \setlength{\itemsep}{0pt}\setlength{\parskip}{0pt}}
\setcounter{secnumdepth}{-\maxdimen} % remove section numbering
\usepackage{float} \usepackage{multicol} \usepackage{amsmath} \usepackage{graphicx} \usepackage{array}

\title{Badges of status and Individual Recognition mediated by learning}
\author{Andrés Quiñones}
\date{September 15, 2020}

\begin{document}
\frame{\titlepage}

\begin{frame}

\end{frame}

\hypertarget{results}{%
\section{Results}\label{results}}

\begin{frame}{Learning under honest signalling}
\protect\hypertarget{learning-under-honest-signalling}{}

Learning simulations (without evolution yet), the sender code is set as
an honest signal. Receivers develop a threshold like response according
to their quality

\begin{center}\includegraphics[width=1\linewidth]{Simulations/alphaAct_/alphaAct0.6learnDyn} \end{center}

\end{frame}

\begin{frame}{Learning can mediate the evolution of Badges of Status}
\protect\hypertarget{learning-can-mediate-the-evolution-of-badges-of-status}{}

\begin{block}{Variation among replicates}

\small

set of simulations with the badge working as a handicap (cost inversely
proportional to quality)

\begin{center}\includegraphics[width=1\linewidth]{Simulations/betCostEvol1_/evolDynAll_betCost5} \end{center}

\end{block}

\end{frame}

\begin{frame}{Learning can mediate the evolution of Badges of Status}
\protect\hypertarget{learning-can-mediate-the-evolution-of-badges-of-status-1}{}

\begin{block}{Individual replicates: Badges}

\small

One of the replicates of the previous slide, the badge as a handicap

\begin{center}\includegraphics[width=1\linewidth]{Simulations/betCostEvol1_/evolDyn0_betCost5} \end{center}

\end{block}

\end{frame}

\begin{frame}{Learning can mediate the evolution of Badges of Status}
\protect\hypertarget{learning-can-mediate-the-evolution-of-badges-of-status-2}{}

\begin{block}{Individual replicates: Badges}

\small

Panels show the changes in frequency distribution along evolution
(\(\alpha\), \(\beta\), and resulting badge respectively) for the
replicate shown in the previous slide

\begin{figure}

\includegraphics[height=0.28\textheight]{Simulations/betCostEvol1_/evolDistAlpha0_betCost5} \includegraphics[height=0.28\textheight]{Simulations/betCostEvol1_/evolDistBeta0_betCost5} \includegraphics[height=0.28\textheight]{Simulations/betCostEvol1_/evolDistBadge0_betCost5} \hfill{}

\caption{alpha - beta -Badge}\label{fig:unnamed-chunk-4}
\end{figure}

\end{block}

\end{frame}

\begin{frame}{Learning can mediate the evolution of Badges of Status}
\protect\hypertarget{learning-can-mediate-the-evolution-of-badges-of-status-3}{}

\begin{block}{Individual replicates: no badges}

\small

One of the replicates where the badge did not evolve

\begin{center}\includegraphics[width=1\linewidth]{Simulations/betCostEvol1_/evolDyn5_betCost5} \end{center}

\end{block}

\end{frame}

\begin{frame}{The effect of costs on Badges of Status}
\protect\hypertarget{the-effect-of-costs-on-badges-of-status}{}

\begin{block}{Behavioural interactions}

\small

Colours correspond to replicates, each point is one snapshot on the
second half of the simulations. Higher costs seem to favour the
signalling equilibrium

\begin{center}\includegraphics[width=0.8\linewidth]{Simulations/betCostEvol1_/BehavIntALL} \end{center}

\end{block}

\end{frame}

\begin{frame}{The effect of costs on Badges of Status}
\protect\hypertarget{the-effect-of-costs-on-badges-of-status-1}{}

\begin{block}{Badge size}

\small

Same as previous slide but for the resulting signal. Higher costs yield
smaller averga sizes and smaller ranges.

\begin{center}\includegraphics[width=0.8\linewidth]{Simulations/betCostEvol1_/CueALL} \end{center}

\end{block}

\end{frame}

\begin{frame}{The evolution of cheap Badges of Status}
\protect\hypertarget{the-evolution-of-cheap-badges-of-status}{}

\begin{block}{Variation among replicates}

\small

Here simulations where the signal has no cost

\begin{center}\includegraphics[width=0.8\linewidth]{Simulations/nIntGroupEvol1_/evolDynALL_nIntGroup2000} \end{center}

\end{block}

\end{frame}

\begin{frame}{The evolution of cheap Badges of Status}
\protect\hypertarget{the-evolution-of-cheap-badges-of-status-1}{}

\begin{block}{Variation within a replicate}

\small

Here simulations where the signal has no cost.

There seems to be some consistency in the three type of sender codes

\begin{center}\includegraphics[width=0.8\linewidth]{Simulations/nIntGroupEvol1_/evolDyn8_nIntGroup2000} \end{center}

\end{block}

\end{frame}

\begin{frame}{The evolution of cheap Badges of Status}
\protect\hypertarget{the-evolution-of-cheap-badges-of-status-2}{}

\begin{block}{Variation within a replicate}

\tiny

Panels show the changes in frequency distribution along evolution
(\(\alpha\), \(\beta\), and resulting badge respectively) for the
replicate shown in the previous slide

Branching events in the reaction norm parameters yield three types of
sender codes. One (more or less) honest, two dishonest (one with a big
badge and one with a small badge)

\begin{figure}

\includegraphics[height=0.28\textheight]{Simulations/nIntGroupEvol1_/evolDistAlpha8_nIntGroup2000} \includegraphics[height=0.28\textheight]{Simulations/nIntGroupEvol1_/evolDistBeta8_nIntGroup2000} \includegraphics[height=0.28\textheight]{Simulations/nIntGroupEvol1_/evolDistBadge8_nIntGroup2000} \hfill{}

\caption{alpha - beta -Badge}\label{fig:unnamed-chunk-10}
\end{figure}

\end{block}

\end{frame}

\begin{frame}{The evolution of cheap Badges of Status}
\protect\hypertarget{the-evolution-of-cheap-badges-of-status-3}{}

\begin{block}{How often does it happen?}

\tiny

Values of \(\alpha\) and \(\beta\) for each individual in the different
replicates. Each dot an individual, each panel a replicate. Individuals
are sampled from populations from the second half of the simulations.

Individuals in some simulations split up in three groups, others in two
groups; in one simulation the population does not split up

\begin{flushleft}\includegraphics[width=0.9\linewidth]{Simulations/nIntGroupEvol1_/corrAlphBet_nIntGroup2000} \end{flushleft}

\end{block}

\end{frame}

\begin{frame}{The evolution of cheap Badges of Status}
\protect\hypertarget{the-evolution-of-cheap-badges-of-status-4}{}

\begin{block}{Does it matter?}

\small

Behavioural interactions splitting up the replicates according to the
number of ``types'' that evolve

\begin{flushleft}\includegraphics[width=0.9\linewidth]{Simulations/nIntGroupEvol1_/BehavIntAllClusters_nIntGroup2000} \end{flushleft}

\end{block}

\end{frame}

\begin{frame}{Interaction structures - do they matter?}
\protect\hypertarget{interaction-structures---do-they-matter}{}

\begin{center}\includegraphics[width=1\linewidth]{Images/networks} \end{center}

(Chaine et al. 2018)

\end{frame}

\begin{frame}{Localized interaction structures}
\protect\hypertarget{localized-interaction-structures}{}

\begin{block}{Costly signals}

\tiny

Behavioural interactions and badge size for simulations with costs where
individuals interact in small groups. On the \(x\) axes number of
different individuals each focal interacts with.

\begin{center}\includegraphics[width=0.9\linewidth]{Simulations/nIntGroupCost_/BehavIntALL} \includegraphics[width=0.9\linewidth]{Simulations/nIntGroupCost_/CueALL} \end{center}

\end{block}

\end{frame}

\begin{frame}{Localized interaction structures}
\protect\hypertarget{localized-interaction-structures-1}{}

\begin{block}{Cheap signals (free)}

\small

Variation among individuals in different replicates (just like before)
in replicates where individuals interact with 10 other individuals.

A part from replicates with two clusters, the clustering seems less
straightforward as before

\begin{center}\includegraphics[width=0.85\linewidth]{Simulations/nIntGroupEvol1_/corrAlphBet_nIntGroup10} \end{center}

\end{block}

\end{frame}

\begin{frame}{Localized interaction structures}
\protect\hypertarget{localized-interaction-structures-2}{}

\begin{block}{Cheap signals (free)}

\small

Behavioural interactions for the replicates form previous slide. I split
up the replicates according to clusters (despite clusters not being
clear cut).

Only replicates with only one cluster seems to have less aggressive
interactions

\begin{center}\includegraphics[width=0.9\linewidth]{Simulations/nIntGroupEvol1_/BehavIntAllClusters_nIntGroup10} \end{center}

\end{block}

\end{frame}

\begin{frame}{Localized interaction structures}
\protect\hypertarget{localized-interaction-structures-3}{}

\begin{block}{Cheap signals or Individual recognition?}

\tiny

In these set, I changed the learning set up, I increased the number of
centers (10 now, 6 before) along the signal axis; and I reduced the
level or generalization.

There seems to be an effect on the slope of the reaction norm. It stays
very close to zero.

\begin{center}\includegraphics[width=0.9\linewidth]{Simulations/nIntGroupCenters1_/evolDynALL_nIntGroup10} \end{center}

\end{block}

\end{frame}

\begin{frame}{Localized interaction structures}
\protect\hypertarget{localized-interaction-structures-4}{}

\begin{block}{Cheap signals or Individual recognition}

\small

Same replicates as the previous slide. Variation in \(\beta\) and
\(\alpha\) among individuals for different replicates.

No clear clustering. Variation mostly along the \(\alpha\) axis

\begin{center}\includegraphics[width=0.9\linewidth]{Simulations/nIntGroupCenters1_/corrAlphBet_nIntGroup10} \end{center}

\end{block}

\end{frame}

\begin{frame}{Localized interaction structures}
\protect\hypertarget{localized-interaction-structures-5}{}

\begin{block}{Cheap signals or Individual recognition?}

\small

Panels show the changes in frequency distribution along evolution
(\(\alpha\), \(\beta\), and resulting badge respectively) for the
replicate shown in the previous slide.

\(\beta\) around zero, \(\alpha\) all over the place.

\begin{flushleft}\includegraphics[height=0.28\textheight]{Simulations/nIntGroupCenters1_/evolDistAlpha5_nIntGroup10} \includegraphics[height=0.28\textheight]{Simulations/nIntGroupCenters1_/evolDistBeta5_nIntGroup10} \includegraphics[height=0.28\textheight]{Simulations/nIntGroupCenters1_/evolDistBadge5_nIntGroup10} \end{flushleft}

\end{block}

\end{frame}

\begin{frame}{Why would you want to be recognized?}
\protect\hypertarget{why-would-you-want-to-be-recognized}{}

\pause

\begin{block}{The peaceful, the aggressive and the clever}

\begin{itemize}
\tightlist
\item
  Learning initial conditions
\end{itemize}

\tiny

Here, I vary the aggressive tendency for unlearned individuals in these
three set ups.

Agressive, peaceful, clever (starts at the ESS of the classic hawk-dove
game)

\begin{flushleft}\includegraphics[width=0.9\linewidth]{Images/cartoonRBF_InitAct} \end{flushleft}

\end{block}

\end{frame}

\begin{frame}{Why would you want to be recognized?}
\protect\hypertarget{why-would-you-want-to-be-recognized-1}{}

\begin{block}{The peaceful}

Variation among replicates

\begin{flushleft}\includegraphics[width=1\linewidth]{Simulations/initAct_/evolDynALL_initAct2} \end{flushleft}

\end{block}

\end{frame}

\begin{frame}{Why would you want to be recognized?}
\protect\hypertarget{why-would-you-want-to-be-recognized-2}{}

\begin{block}{The peaceful}

Variation within one replicate

\begin{flushleft}\includegraphics[width=1\linewidth]{Simulations/initAct_/evolDyn2_initAct2} \end{flushleft}

\end{block}

\end{frame}

\begin{frame}{Why would you want to be recognized?}
\protect\hypertarget{why-would-you-want-to-be-recognized-3}{}

\begin{block}{The peaceful}

Changes in the frequency distributions

\begin{flushleft}\includegraphics[height=0.28\textheight]{Simulations/initAct_/evolDistAlpha2_initAct2} \includegraphics[height=0.28\textheight]{Simulations/initAct_/evolDistBeta2_initAct2} \includegraphics[height=0.28\textheight]{Simulations/initAct_/evolDistBadge2_initAct2} \end{flushleft}

\end{block}

\end{frame}

\begin{frame}{Why would you want to be recognized?}
\protect\hypertarget{why-would-you-want-to-be-recognized-4}{}

\begin{block}{The peaceful}

Favours variation and cheap signals

\begin{flushleft}\includegraphics[width=1\linewidth]{Simulations/initAct_/corrAlphBet_initAct2} \end{flushleft}

\end{block}

\end{frame}

\begin{frame}{Why would you want to be recognized?}
\protect\hypertarget{why-would-you-want-to-be-recognized-5}{}

\begin{block}{The aggressive}

Variation among replicates

\begin{flushleft}\includegraphics[width=1\linewidth]{Simulations/initAct_/evolDynALL_initAct-2} \end{flushleft}

\end{block}

\end{frame}

\begin{frame}{Why would you want to be recognized?}
\protect\hypertarget{why-would-you-want-to-be-recognized-6}{}

\begin{block}{The aggressive}

Variation within one replicate

\begin{flushleft}\includegraphics[width=1\linewidth]{Simulations/initAct_/evolDyn0_initAct-2} \end{flushleft}

\end{block}

\end{frame}

\begin{frame}{Why would you want to be recognized?}
\protect\hypertarget{why-would-you-want-to-be-recognized-7}{}

\begin{block}{The aggressive}

Disfavours variation and signal evolution - no signalling

\begin{flushleft}\includegraphics[width=1\linewidth]{Simulations/initAct_/corrAlphBet_initAct-2} \end{flushleft}

\end{block}

\end{frame}

\begin{frame}{Why would you want to be recognized?}
\protect\hypertarget{why-would-you-want-to-be-recognized-8}{}

\begin{block}{The clever (ESS)}

Variation among replicates

\begin{flushleft}\includegraphics[width=1\linewidth]{Simulations/initAct_/evolDynALL_initAct-0.69} \end{flushleft}

\end{block}

\end{frame}

\begin{frame}{Why would you want to be recognized?}
\protect\hypertarget{why-would-you-want-to-be-recognized-9}{}

\begin{block}{The clever (ESS)}

Variation within one replicate

\begin{flushleft}\includegraphics[width=1\linewidth]{Simulations/initAct_/evolDyn0_initAct-0.69} \end{flushleft}

\end{block}

\end{frame}

\begin{frame}{Why would you want to be recognized?}
\protect\hypertarget{why-would-you-want-to-be-recognized-10}{}

\begin{block}{The clever (ESS)}

Signal evolution seems to be driven by drift. Most replicates end up in
one of the two extremes

\begin{flushleft}\includegraphics[width=1\linewidth]{Simulations/initAct_/corrAlphBet_initAct-0.69} \end{flushleft}

\end{block}

\end{frame}

\begin{frame}{Take-home messages}
\protect\hypertarget{take-home-messages}{}

\begin{block}{Costly signals}

\begin{itemize}
\tightlist
\item
  Learning can mediate costly signals
\item
  More expensive signals seem be more likely to evolve
\item
  Less expensive signals provide more information
\item
  Local interactions do not impede the evolution of costly signals (as
  has been suggested)
\end{itemize}

\end{block}

\begin{block}{Cheap signals}

\begin{itemize}
\tightlist
\item
  Learning mediates the evolution of polymorphisms with ``honest'' and
  ``dishonest'' individuals
\item
  Under local interactions learning seems to allow the evolution of IR
\item
  Cheap signals depend on the innate level of aggression (before
  learning) of individuals
\item
  Peace begets heterogeneity, aggression begets homogeneity
\end{itemize}

\end{block}

\end{frame}

\begin{frame}{Questions?}
\protect\hypertarget{questions}{}

\begin{center}\includegraphics[height=0.8\textheight]{Images/BizarroParrot} \end{center}

\end{frame}

\begin{frame}{References}
\protect\hypertarget{references}{}

\hypertarget{refs}{}
\leavevmode\hypertarget{ref-chaine_Manipulating_2018}{}%
Chaine, Alexis S., Daizaburo Shizuka, Theadora A. Block, Lynn Zhang, and
Bruce E. Lyon. 2018. ``Manipulating Badges of Status Only Fools
Strangers.'' \emph{Ecology Letters} 21 (10): 1477--85.
\url{https://doi.org/10.1111/ele.13128}.

\end{frame}

\end{document}
