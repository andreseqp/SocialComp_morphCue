% Options for packages loaded elsewhere
\PassOptionsToPackage{unicode}{hyperref}
\PassOptionsToPackage{hyphens}{url}
%
\documentclass[
  ignorenonframetext,
]{beamer}
\usepackage{pgfpages}
\setbeamertemplate{caption}[numbered]
\setbeamertemplate{caption label separator}{: }
\setbeamercolor{caption name}{fg=normal text.fg}
\beamertemplatenavigationsymbolsempty
% Prevent slide breaks in the middle of a paragraph
\widowpenalties 1 10000
\raggedbottom
\setbeamertemplate{part page}{
  \centering
  \begin{beamercolorbox}[sep=16pt,center]{part title}
    \usebeamerfont{part title}\insertpart\par
  \end{beamercolorbox}
}
\setbeamertemplate{section page}{
  \centering
  \begin{beamercolorbox}[sep=12pt,center]{part title}
    \usebeamerfont{section title}\insertsection\par
  \end{beamercolorbox}
}
\setbeamertemplate{subsection page}{
  \centering
  \begin{beamercolorbox}[sep=8pt,center]{part title}
    \usebeamerfont{subsection title}\insertsubsection\par
  \end{beamercolorbox}
}
\AtBeginPart{
  \frame{\partpage}
}
\AtBeginSection{
  \ifbibliography
  \else
    \frame{\sectionpage}
  \fi
}
\AtBeginSubsection{
  \frame{\subsectionpage}
}
\usepackage{lmodern}
\usepackage{amssymb,amsmath}
\usepackage{ifxetex,ifluatex}
\ifnum 0\ifxetex 1\fi\ifluatex 1\fi=0 % if pdftex
  \usepackage[T1]{fontenc}
  \usepackage[utf8]{inputenc}
  \usepackage{textcomp} % provide euro and other symbols
\else % if luatex or xetex
  \usepackage{unicode-math}
  \defaultfontfeatures{Scale=MatchLowercase}
  \defaultfontfeatures[\rmfamily]{Ligatures=TeX,Scale=1}
\fi
% Use upquote if available, for straight quotes in verbatim environments
\IfFileExists{upquote.sty}{\usepackage{upquote}}{}
\IfFileExists{microtype.sty}{% use microtype if available
  \usepackage[]{microtype}
  \UseMicrotypeSet[protrusion]{basicmath} % disable protrusion for tt fonts
}{}
\makeatletter
\@ifundefined{KOMAClassName}{% if non-KOMA class
  \IfFileExists{parskip.sty}{%
    \usepackage{parskip}
  }{% else
    \setlength{\parindent}{0pt}
    \setlength{\parskip}{6pt plus 2pt minus 1pt}}
}{% if KOMA class
  \KOMAoptions{parskip=half}}
\makeatother
\usepackage{xcolor}
\IfFileExists{xurl.sty}{\usepackage{xurl}}{} % add URL line breaks if available
\IfFileExists{bookmark.sty}{\usepackage{bookmark}}{\usepackage{hyperref}}
\hypersetup{
  pdfauthor={Andrés Quiñones PhD.},
  hidelinks,
  pdfcreator={LaTeX via pandoc}}
\urlstyle{same} % disable monospaced font for URLs
\newif\ifbibliography
\setlength{\emergencystretch}{3em} % prevent overfull lines
\providecommand{\tightlist}{%
  \setlength{\itemsep}{0pt}\setlength{\parskip}{0pt}}
\setcounter{secnumdepth}{-\maxdimen} % remove section numbering
\usepackage{float} \usepackage{multicol} \usepackage{amsmath} \usepackage{graphicx} \usepackage{array} \usepackage{xcolor}

\title{Learning can mediate the evolution of\\
(cheap and costly) signals of quality}
\author{Andrés Quiñones PhD.}
\date{Genes as environment - AGA2020 virtual symposium\\
November 02, 2020}

\begin{document}
\frame{\titlepage}

\begin{frame}{Social interactions}
\protect\hypertarget{social-interactions}{}

\begin{block}{The need to match partners with actions}

\begin{center}\includegraphics[width=0.4\linewidth]{../Images/sexSelec} \includegraphics[width=0.4\linewidth]{../Images/begging} \includegraphics[width=0.4\linewidth]{../Images/broodparasite} \includegraphics[width=0.4\linewidth]{../Images/dominance} \end{center}

\end{block}

\end{frame}

\begin{frame}{The effect of communication}
\protect\hypertarget{the-effect-of-communication}{}

\begin{block}{Badge of status}

\begin{center}\includegraphics[height=0.8\textheight]{../Images//hosp_alexandra_mackenzie} \end{center}

\end{block}

\end{frame}

\begin{frame}{What about signals?}
\protect\hypertarget{what-about-signals}{}

\begin{block}{The handicap principle}

\begin{center}\includegraphics[height=0.35\textheight]{../Images/Amotz_Zahavi_profile} \includegraphics[height=0.35\textheight]{../Images/sexualSelection} \includegraphics[height=0.35\textheight]{../Images/Grafen} \end{center}

\begin{center}\includegraphics[height=0.35\textheight]{GenesAsEnviron_files/figure-beamer/unnamed-chunk-4-1} \end{center}

(Zahavi 1975; Grafen 1990)

\end{block}

\end{frame}

\begin{frame}{The handicap principle}
\protect\hypertarget{the-handicap-principle-1}{}

\begin{block}{Badges of status}

\begin{center}\includegraphics[width=0.9\linewidth]{../Images//comSystemBot} \end{center}

(Botero et al. 2010)

\end{block}

\end{frame}

\begin{frame}{Are bibs really badges?}
\protect\hypertarget{are-bibs-really-badges}{}

\begin{flushleft}\includegraphics[width=0.3\linewidth]{../Images/Sanchez-Tojar_2018} \end{flushleft}

\end{frame}

\begin{frame}{What about learning?}
\protect\hypertarget{what-about-learning}{}

\begin{center}\includegraphics[height=0.4\textheight]{../Images//pavlov} \end{center}

\begin{center}\includegraphics[height=0.35\textheight]{../Images/learn_cartoon_0} \end{center}

\end{frame}

\begin{frame}{What about learning?}
\protect\hypertarget{what-about-learning-1}{}

\begin{center}\includegraphics[height=0.4\textheight]{../Images//pavlov} \end{center}

\begin{center}\includegraphics[height=0.35\textheight]{../Images/learn_cartoon_1} \end{center}

\end{frame}

\begin{frame}{What about learning?}
\protect\hypertarget{what-about-learning-2}{}

\begin{center}\includegraphics[height=0.4\textheight]{../Images//pavlov} \end{center}

\begin{center}\includegraphics[height=0.35\textheight]{../Images/learn_cartoon_2} \end{center}

\end{frame}

\begin{frame}{What about learning?}
\protect\hypertarget{what-about-learning-3}{}

\begin{center}\includegraphics[height=0.4\textheight]{../Images//pavlov} \end{center}

\begin{center}\includegraphics[height=0.35\textheight]{../Images/learn_cartoon_3} \end{center}

\end{frame}

\begin{frame}{What about learning?}
\protect\hypertarget{what-about-learning-4}{}

\begin{center}\includegraphics[height=0.4\textheight]{../Images//pavlov} \end{center}

\begin{center}\includegraphics[height=0.35\textheight]{../Images/learn_cartoon_5} \end{center}

\end{frame}

\begin{frame}{What about learning?}
\protect\hypertarget{what-about-learning-5}{}

\begin{center}\includegraphics[height=0.4\textheight]{../Images//pavlov} \end{center}

\begin{center}\includegraphics[height=0.35\textheight]{../Images/learn_cartoon_6} \end{center}

\end{frame}

\begin{frame}{A model of communication in animal conflict}
\protect\hypertarget{a-model-of-communication-in-animal-conflict}{}

\vspace{-0.5cm}

\begin{block}{Comunication}

\begin{columns}[T]

\column{0.45\linewidth}

The sender code

\tiny
\begin{equation*}
B_i = \frac{1}{1+e^{-(\textcolor{red}{\alpha_s} +\textcolor{red}{\beta_s} Q_i)}}
\end{equation*}



\begin{center}\includegraphics[width=0.5\linewidth]{GenesAsEnviron_files/figure-beamer/unnamed-chunk-19-1} \end{center}


\column{0.45\linewidth}

The receiver code (learning)

\tiny
\begin{equation*}
  \Delta V_{t(s)}=\alpha (R_t-V_t)
\end{equation*}



\begin{center}\includegraphics[width=0.6\linewidth]{GenesAsEnviron_files/figure-beamer/unnamed-chunk-20-1} \end{center}

\end{columns}

\end{block}

\begin{block}{Conflict}

\vspace{0.3cm}
\begin{columns}
\column{0.5\linewidth}
hawk-dove game
\tiny
\begin{center}
\begin{tabular}{ >{\centering\arraybackslash}p{0.2cm} | >{\centering\arraybackslash}p{2.5cm} | >{\centering\arraybackslash}p{0.8cm} }
& H & D \\ \hline
H & $p_w V\frac{-C}{2} + (1-p_w) \frac{-C}{2}$ & $V$ \\ \hline
V & $0$ & $\frac{V}{2}$\\
\end{tabular}
\end{center}

\begin{equation*}
p_w=\frac{1}{1+e^{-\beta(Q_i-Q_j)}}
\end{equation*}

\column{0.5\linewidth}



\begin{center}\includegraphics[width=0.6\linewidth]{GenesAsEnviron_files/figure-beamer/unnamed-chunk-21-1} \end{center}

\end{columns}

\end{block}

\end{frame}

\hypertarget{results}{%
\section{Results}\label{results}}

\begin{frame}{Learning under honest signalling}
\protect\hypertarget{learning-under-honest-signalling}{}

\pause

\begin{center}\includegraphics[width=1\linewidth]{../Simulations/alphaAct_/alphaAct0.6learnDyn} \end{center}

\end{frame}

\begin{frame}{Learning mediates the evolution of badges of status
(handicaps)}
\protect\hypertarget{learning-mediates-the-evolution-of-badges-of-status-handicaps}{}

\begin{center}\includegraphics[width=1\linewidth]{../Simulations/betCostEvol1_/evolDyn0_betCost5} \end{center}

\end{frame}

\begin{frame}{Learning can mediate the evolution of Badges of Status}
\protect\hypertarget{learning-can-mediate-the-evolution-of-badges-of-status}{}

\begin{block}{Individual replicates: no badges}

\begin{center}\includegraphics[width=1\linewidth]{../Simulations/betCostEvol1_/evolDyn5_betCost5} \end{center}

\end{block}

\end{frame}

\begin{frame}{Badges of Status: do they matter?}
\protect\hypertarget{badges-of-status-do-they-matter}{}

\begin{block}{Behavioural interactions}

\begin{center}\includegraphics[width=0.9\linewidth]{../Simulations/betCostEvol1_/compClustRep_betCost5} \end{center}

--\textgreater{} --\textgreater{} --\textgreater{}

\end{block}

\end{frame}

\begin{frame}{The evolution of cheap Badges of Status}
\protect\hypertarget{the-evolution-of-cheap-badges-of-status}{}

\begin{center}\includegraphics[width=0.9\linewidth]{../Simulations/nIntGroupEvol4_/evolDyn13_nIntGroup2000} \end{center}

\end{frame}

\begin{frame}{The evolution of cheap Badges of Status}
\protect\hypertarget{the-evolution-of-cheap-badges-of-status-1}{}

\begin{figure}

\includegraphics[height=0.28\textheight]{../Simulations/nIntGroupEvol4_/evolDistAlpha13_nIntGroup2000} \includegraphics[height=0.28\textheight]{../Simulations/nIntGroupEvol4_/evolDistBeta13_nIntGroup2000} \includegraphics[height=0.28\textheight]{../Simulations/nIntGroupEvol4_/evolDistBadge13_nIntGroup2000} \hfill{}

\caption{alpha - beta -Badge}\label{fig:unnamed-chunk-27}
\end{figure}

\end{frame}

\begin{frame}{The evolution of cheap Badges of Status}
\protect\hypertarget{the-evolution-of-cheap-badges-of-status-2}{}

\begin{block}{How often does it happen?}

\begin{flushleft}\includegraphics[width=0.9\linewidth]{../Simulations/nIntGroupEvol1_/corrAlphBet_nIntGroup2000} \end{flushleft}

\end{block}

\end{frame}

\begin{frame}{The evolution of cheap Badges of Status}
\protect\hypertarget{the-evolution-of-cheap-badges-of-status-3}{}

\begin{block}{Does it matter?}

\small

Behavioural interactions splitting up the replicates according to the
number of ``types'' that evolve

\begin{flushleft}\includegraphics[width=0.9\linewidth]{../Simulations/nIntGroupEvol1_/BehavIntAllClusters_nIntGroup2000} \end{flushleft}

--\textgreater{}

--\textgreater{}

\end{block}

\end{frame}

\begin{frame}{Why would you want to be recognized?}
\protect\hypertarget{why-would-you-want-to-be-recognized}{}

\pause

\begin{block}{The peaceful, the aggressive and the clever}

\begin{itemize}
\tightlist
\item
  Learning initial conditions
\end{itemize}

\tiny

Here, I vary the aggressive tendency for unlearned individuals in these
three set ups.

Agressive, peaceful, clever (starts at the ESS of the classic hawk-dove
game)

\begin{flushleft}\includegraphics[width=0.9\linewidth]{../Images/cartoonRBF_InitAct} \end{flushleft}

\end{block}

\end{frame}

\begin{frame}{Why would you want to be recognized?}
\protect\hypertarget{why-would-you-want-to-be-recognized-1}{}

\begin{block}{The peaceful}

Variation among replicates

\begin{flushleft}\includegraphics[width=1\linewidth]{../Simulations/initAct_/evolDynALL_initAct2} \end{flushleft}

\end{block}

\end{frame}

\begin{frame}{Why would you want to be recognized?}
\protect\hypertarget{why-would-you-want-to-be-recognized-2}{}

\begin{block}{The peaceful}

Variation within one replicate

\begin{flushleft}\includegraphics[width=1\linewidth]{../Simulations/initAct_/evolDyn2_initAct2} \end{flushleft}

\end{block}

\end{frame}

\begin{frame}{Why would you want to be recognized?}
\protect\hypertarget{why-would-you-want-to-be-recognized-3}{}

\begin{block}{The peaceful}

Changes in the frequency distributions

\begin{flushleft}\includegraphics[height=0.28\textheight]{../Simulations/initAct_/evolDistAlpha2_initAct2} \includegraphics[height=0.28\textheight]{../Simulations/initAct_/evolDistBeta2_initAct2} \includegraphics[height=0.28\textheight]{../Simulations/initAct_/evolDistBadge2_initAct2} \end{flushleft}

\end{block}

\end{frame}

\begin{frame}{Why would you want to be recognized?}
\protect\hypertarget{why-would-you-want-to-be-recognized-4}{}

\begin{block}{The peaceful}

Favours variation and cheap signals

\begin{flushleft}\includegraphics[width=1\linewidth]{../Simulations/initAct_/corrAlphBet_initAct2} \end{flushleft}

\end{block}

\end{frame}

\begin{frame}{Why would you want to be recognized?}
\protect\hypertarget{why-would-you-want-to-be-recognized-5}{}

\begin{block}{The aggressive}

Variation among replicates

\begin{flushleft}\includegraphics[width=1\linewidth]{../Simulations/initAct_/evolDynALL_initAct-2} \end{flushleft}

\end{block}

\end{frame}

\begin{frame}{Why would you want to be recognized?}
\protect\hypertarget{why-would-you-want-to-be-recognized-6}{}

\begin{block}{The aggressive}

Variation within one replicate

\begin{flushleft}\includegraphics[width=1\linewidth]{../Simulations/initAct_/evolDyn0_initAct-2} \end{flushleft}

\end{block}

\end{frame}

\begin{frame}{Why would you want to be recognized?}
\protect\hypertarget{why-would-you-want-to-be-recognized-7}{}

\begin{block}{The aggressive}

Disfavours variation and signal evolution - no signalling

\begin{flushleft}\includegraphics[width=1\linewidth]{../Simulations/initAct_/corrAlphBet_initAct-2} \end{flushleft}

\end{block}

\end{frame}

\begin{frame}{Why would you want to be recognized?}
\protect\hypertarget{why-would-you-want-to-be-recognized-8}{}

\begin{block}{The clever (ESS)}

Variation among replicates

\begin{flushleft}\includegraphics[width=1\linewidth]{../Simulations/initAct_/evolDynALL_initAct-0.69} \end{flushleft}

\end{block}

\end{frame}

\begin{frame}{Why would you want to be recognized?}
\protect\hypertarget{why-would-you-want-to-be-recognized-9}{}

\begin{block}{The clever (ESS)}

Variation within one replicate

\begin{flushleft}\includegraphics[width=1\linewidth]{../Simulations/initAct_/evolDyn0_initAct-0.69} \end{flushleft}

\end{block}

\end{frame}

\begin{frame}{Why would you want to be recognized?}
\protect\hypertarget{why-would-you-want-to-be-recognized-10}{}

\begin{block}{The clever (ESS)}

Signal evolution seems to be driven by drift. Most replicates end up in
one of the two extremes

\begin{flushleft}\includegraphics[width=1\linewidth]{../Simulations/initAct_/corrAlphBet_initAct-0.69} \end{flushleft}

\end{block}

\end{frame}

\begin{frame}{Take-home messages}
\protect\hypertarget{take-home-messages}{}

\begin{block}{Costly signals}

\begin{itemize}
\tightlist
\item
  Learning can mediate costly signals
\item
  More expensive signals seem be more likely to evolve
\item
  Less expensive signals provide more information
\item
  Local interactions do not impede the evolution of costly signals (as
  has been suggested)
\end{itemize}

\end{block}

\begin{block}{Cheap signals}

\begin{itemize}
\tightlist
\item
  Learning mediates the evolution of polymorphisms with ``honest'' and
  ``dishonest'' individuals
\item
  Under local interactions learning seems to allow the evolution of IR
\item
  Cheap signals depend on the innate level of aggression (before
  learning) of individuals
\item
  Peace begets heterogeneity, aggression begets homogeneity
\end{itemize}

\end{block}

\end{frame}

\begin{frame}{Questions?}
\protect\hypertarget{questions}{}

\begin{center}\includegraphics[height=0.8\textheight]{../Images/BizarroParrot} \end{center}

\end{frame}

\begin{frame}{References}
\protect\hypertarget{references}{}

\hypertarget{refs}{}
\leavevmode\hypertarget{ref-botero_Evolution_2010}{}%
Botero, C. A, I. Pen, J. Komdeur, and F. J Weissing. 2010. ``The
Evolution of Individual Variation in Communication Strategies.''
\emph{Evolution} 64 (11): 3123--33.

\leavevmode\hypertarget{ref-grafen_Biological_1990}{}%
Grafen, Alan. 1990. ``Biological Signals as Handicaps.'' \emph{Journal
of Theoretical Biology} 144 (4): 517--46.
\url{https://doi.org/10.1016/S0022-5193(05)80088-8}.

\leavevmode\hypertarget{ref-zahavi_Mate_1975}{}%
Zahavi, Amotz. 1975. ``Mate Selection-A Selection for a Handicap.''
\emph{Journal of Theoretical Biology} 53 (1): 205--14.
\url{https://doi.org/10.1016/0022-5193(75)90111-3}.

\end{frame}

\end{document}
