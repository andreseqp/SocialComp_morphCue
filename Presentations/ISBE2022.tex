% Options for packages loaded elsewhere
\PassOptionsToPackage{unicode}{hyperref}
\PassOptionsToPackage{hyphens}{url}
%
\documentclass[
  ignorenonframetext,
]{beamer}
\usepackage{pgfpages}
\setbeamertemplate{caption}[numbered]
\setbeamertemplate{caption label separator}{: }
\setbeamercolor{caption name}{fg=normal text.fg}
\beamertemplatenavigationsymbolsempty
% Prevent slide breaks in the middle of a paragraph
\widowpenalties 1 10000
\raggedbottom
\setbeamertemplate{part page}{
  \centering
  \begin{beamercolorbox}[sep=16pt,center]{part title}
    \usebeamerfont{part title}\insertpart\par
  \end{beamercolorbox}
}
\setbeamertemplate{section page}{
  \centering
  \begin{beamercolorbox}[sep=12pt,center]{part title}
    \usebeamerfont{section title}\insertsection\par
  \end{beamercolorbox}
}
\setbeamertemplate{subsection page}{
  \centering
  \begin{beamercolorbox}[sep=8pt,center]{part title}
    \usebeamerfont{subsection title}\insertsubsection\par
  \end{beamercolorbox}
}
\AtBeginPart{
  \frame{\partpage}
}
\AtBeginSection{
  \ifbibliography
  \else
    \frame{\sectionpage}
  \fi
}
\AtBeginSubsection{
  \frame{\subsectionpage}
}
\usepackage{amsmath,amssymb}
\usepackage{lmodern}
\usepackage{iftex}
\ifPDFTeX
  \usepackage[T1]{fontenc}
  \usepackage[utf8]{inputenc}
  \usepackage{textcomp} % provide euro and other symbols
\else % if luatex or xetex
  \usepackage{unicode-math}
  \defaultfontfeatures{Scale=MatchLowercase}
  \defaultfontfeatures[\rmfamily]{Ligatures=TeX,Scale=1}
\fi
% Use upquote if available, for straight quotes in verbatim environments
\IfFileExists{upquote.sty}{\usepackage{upquote}}{}
\IfFileExists{microtype.sty}{% use microtype if available
  \usepackage[]{microtype}
  \UseMicrotypeSet[protrusion]{basicmath} % disable protrusion for tt fonts
}{}
\makeatletter
\@ifundefined{KOMAClassName}{% if non-KOMA class
  \IfFileExists{parskip.sty}{%
    \usepackage{parskip}
  }{% else
    \setlength{\parindent}{0pt}
    \setlength{\parskip}{6pt plus 2pt minus 1pt}}
}{% if KOMA class
  \KOMAoptions{parskip=half}}
\makeatother
\usepackage{xcolor}
\IfFileExists{xurl.sty}{\usepackage{xurl}}{} % add URL line breaks if available
\IfFileExists{bookmark.sty}{\usepackage{bookmark}}{\usepackage{hyperref}}
\hypersetup{
  pdfauthor={Andrés Quiñones},
  hidelinks,
  pdfcreator={LaTeX via pandoc}}
\urlstyle{same} % disable monospaced font for URLs
\newif\ifbibliography
\usepackage{longtable,booktabs,array}
\usepackage{calc} % for calculating minipage widths
\usepackage{caption}
% Make caption package work with longtable
\makeatletter
\def\fnum@table{\tablename~\thetable}
\makeatother
\usepackage{graphicx}
\makeatletter
\def\maxwidth{\ifdim\Gin@nat@width>\linewidth\linewidth\else\Gin@nat@width\fi}
\def\maxheight{\ifdim\Gin@nat@height>\textheight\textheight\else\Gin@nat@height\fi}
\makeatother
% Scale images if necessary, so that they will not overflow the page
% margins by default, and it is still possible to overwrite the defaults
% using explicit options in \includegraphics[width, height, ...]{}
\setkeys{Gin}{width=\maxwidth,height=\maxheight,keepaspectratio}
% Set default figure placement to htbp
\makeatletter
\def\fps@figure{htbp}
\makeatother
\setlength{\emergencystretch}{3em} % prevent overfull lines
\providecommand{\tightlist}{%
  \setlength{\itemsep}{0pt}\setlength{\parskip}{0pt}}
\setcounter{secnumdepth}{-\maxdimen} % remove section numbering
\newlength{\cslhangindent}
\setlength{\cslhangindent}{1.5em}
\newlength{\csllabelwidth}
\setlength{\csllabelwidth}{3em}
\newlength{\cslentryspacingunit} % times entry-spacing
\setlength{\cslentryspacingunit}{\parskip}
\newenvironment{CSLReferences}[2] % #1 hanging-ident, #2 entry spacing
 {% don't indent paragraphs
  \setlength{\parindent}{0pt}
  % turn on hanging indent if param 1 is 1
  \ifodd #1
  \let\oldpar\par
  \def\par{\hangindent=\cslhangindent\oldpar}
  \fi
  % set entry spacing
  \setlength{\parskip}{#2\cslentryspacingunit}
 }%
 {}
\usepackage{calc}
\newcommand{\CSLBlock}[1]{#1\hfill\break}
\newcommand{\CSLLeftMargin}[1]{\parbox[t]{\csllabelwidth}{#1}}
\newcommand{\CSLRightInline}[1]{\parbox[t]{\linewidth - \csllabelwidth}{#1}\break}
\newcommand{\CSLIndent}[1]{\hspace{\cslhangindent}#1}
\usepackage{float} \usepackage{multicol} \usepackage{amsmath} \usepackage{graphicx} \usepackage{array} \usepackage{xcolor}
\ifLuaTeX
  \usepackage{selnolig}  % disable illegal ligatures
\fi

\title{Learning can mediate the evolution of\\
(cheap and costly) signals of quality}
\author{Andrés Quiñones}
\date{\hfill\break
\hfill\break
ISBE - Stockholm 2022\\
August 2 - 2020}

\begin{document}
\frame{\titlepage}

\begin{frame}{Acknowledgements}
\protect\hypertarget{acknowledgements}{}
\begin{longtable}[]{@{}
  >{\centering\arraybackslash}p{(\columnwidth - 2\tabcolsep) * \real{0.5000}}
  >{\centering\arraybackslash}p{(\columnwidth - 2\tabcolsep) * \real{0.5000}}@{}}
\toprule
\begin{minipage}[b]{\linewidth}\centering
\includegraphics[width=0.4\textwidth,height=\textheight]{../Images/redouan.jpg}
\end{minipage} & \begin{minipage}[b]{\linewidth}\centering
\includegraphics[width=0.55\textwidth,height=\textheight]{../Images/danielcadena.jpg}
\end{minipage} \\
\midrule
\endhead
\emph{R. Bshary} & \emph{D. Cadena} \\
\includegraphics[width=0.3\textwidth,height=\textheight]{../Images/UNINE_cmjn.jpg}
&
\includegraphics[width=0.3\textwidth,height=\textheight]{../Images/logo_largo.png} \\
\bottomrule
\end{longtable}
\end{frame}

\begin{frame}{Social interactions}
\protect\hypertarget{social-interactions}{}
\begin{block}{The need to match partners with actions}
\protect\hypertarget{the-need-to-match-partners-with-actions}{}
\begin{center}\includegraphics[width=0.4\linewidth]{../Images/sexSelec} \end{center}

\begin{center}\includegraphics[width=0.4\linewidth]{../Images/dominance} \end{center}
\end{block}
\end{frame}

\begin{frame}{The effect of communication}
\protect\hypertarget{the-effect-of-communication}{}
\begin{block}{Badge of status}
\protect\hypertarget{badge-of-status}{}
\begin{center}\includegraphics[width=0.45\linewidth]{../Images/HouseSparrow_strong} \includegraphics[width=0.45\linewidth]{../Images/HouseSparrow_weak} \end{center}
\end{block}
\end{frame}

\begin{frame}{What about signals?}
\protect\hypertarget{what-about-signals}{}
\begin{block}{The handicap principle}
\protect\hypertarget{the-handicap-principle}{}
\begin{center}\includegraphics[height=0.35\textheight]{../Images/Amotz_Zahavi_profile} \includegraphics[height=0.35\textheight]{../Images/sexualSelection} \includegraphics[height=0.35\textheight]{../Images/Grafen} \end{center}

\begin{center}\includegraphics[height=0.35\textheight]{ISBE2022_files/figure-beamer/unnamed-chunk-4-1} \end{center}

(Zahavi 1975; Grafen 1990)
\end{block}
\end{frame}

\begin{frame}{What about learning?}
\protect\hypertarget{what-about-learning}{}
\begin{center}\includegraphics[height=0.4\textheight]{../Images//pavlov} \end{center}

\begin{center}\includegraphics[width=16.76in,height=0.35\textheight]{../Images/learn_cartoon_0} \end{center}
\end{frame}

\begin{frame}{What about learning?}
\protect\hypertarget{what-about-learning-1}{}
\begin{center}\includegraphics[height=0.4\textheight]{../Images//pavlov} \end{center}

\begin{center}\includegraphics[width=16.76in,height=0.35\textheight]{../Images/learn_cartoon_1} \end{center}
\end{frame}

\begin{frame}{What about learning?}
\protect\hypertarget{what-about-learning-2}{}
\begin{center}\includegraphics[height=0.4\textheight]{../Images//pavlov} \end{center}

\begin{center}\includegraphics[width=16.76in,height=0.35\textheight]{../Images/learn_cartoon_2} \end{center}
\end{frame}

\begin{frame}{What about learning?}
\protect\hypertarget{what-about-learning-3}{}
\begin{center}\includegraphics[height=0.4\textheight]{../Images//pavlov} \end{center}

\begin{center}\includegraphics[width=16.76in,height=0.35\textheight]{../Images/learn_cartoon_3} \end{center}
\end{frame}

\begin{frame}{What about learning?}
\protect\hypertarget{what-about-learning-4}{}
\begin{center}\includegraphics[height=0.4\textheight]{../Images//pavlov} \end{center}

\begin{center}\includegraphics[width=16.76in,height=0.35\textheight]{../Images/learn_cartoon_5} \end{center}
\end{frame}

\begin{frame}{What about learning?}
\protect\hypertarget{what-about-learning-5}{}
\begin{center}\includegraphics[height=0.4\textheight]{../Images//pavlov} \end{center}

\begin{center}\includegraphics[width=16.76in,height=0.35\textheight]{../Images/learn_cartoon_6} \end{center}
\end{frame}

\begin{frame}{A model of communication in animal conflict}
\protect\hypertarget{a-model-of-communication-in-animal-conflict}{}
\vspace{-0.3cm}

\begin{block}{Conflict}
\protect\hypertarget{conflict}{}
\vspace{0.3cm}
\begin{columns}
\column{0.5\linewidth}
hawk-dove game
\tiny
\begin{center}
\begin{tabular}{ >{\centering\arraybackslash}p{0.2cm} | >{\centering\arraybackslash}p{2.5cm} | >{\centering\arraybackslash}p{0.8cm} }
& H & D \\ \hline
H & $p_w V\frac{-C}{2} + (1-p_w) \frac{-C}{2}$ & $V$ \\ \hline
V & $0$ & $\frac{V}{2}$\\
\end{tabular}
\end{center}

\begin{equation*}
p_w=\frac{1}{1+e^{-\beta(Q_i-Q_j)}}
\end{equation*}

\column{0.5\linewidth}



\begin{center}\includegraphics[width=0.6\linewidth]{ISBE2022_files/figure-beamer/unnamed-chunk-17-1} \end{center}

\end{columns}

\pause
\end{block}

\begin{block}{Communication}
\protect\hypertarget{communication}{}
\begin{columns}[T]


\column{0.45\linewidth}

The sender code

\tiny
\begin{equation*}
B_i = \frac{1}{1+e^{-(\textcolor{red}{\alpha_s} +\textcolor{red}{\beta_s} Q_i)}}
\end{equation*}



\begin{center}\includegraphics[width=0.5\linewidth]{ISBE2022_files/figure-beamer/unnamed-chunk-18-1} \end{center}


\column{0.45\linewidth}

The receiver code (learning)

\tiny
\begin{equation*}
  \Delta V_{t(s)}=\alpha (R_t-V_t)
\end{equation*}



\begin{center}\includegraphics[width=0.6\linewidth]{ISBE2022_files/figure-beamer/unnamed-chunk-19-1} \end{center}

\end{columns}
\end{block}
\end{frame}

\begin{frame}{Learning under honest signalling}
\protect\hypertarget{learning-under-honest-signalling}{}
\begin{center}\includegraphics[width=1\linewidth]{../Simulations/alphaAct_/alphaAct0.4learnDyn} \end{center}
\end{frame}

\hypertarget{results}{%
\section{Results}\label{results}}

\begin{frame}{Learning mediates the evolution of badges of status
(handicaps)}
\protect\hypertarget{learning-mediates-the-evolution-of-badges-of-status-handicaps}{}
\begin{center}\includegraphics[width=0.9\linewidth]{../Simulations/betCostEvol1_/evolDyn0_betCost5} \end{center}
\end{frame}

\begin{frame}{Learning can mediate the evolution of Badges of Status}
\protect\hypertarget{learning-can-mediate-the-evolution-of-badges-of-status}{}
\begin{block}{Individual replicates: no badges}
\protect\hypertarget{individual-replicates-no-badges}{}
\begin{center}\includegraphics[width=19.44in,height=0.9\textheight]{../Simulations/betCostEvol1_/evolDyn5_betCost5} \end{center}
\end{block}
\end{frame}

\begin{frame}{The evolution of cheap badges of Status}
\protect\hypertarget{the-evolution-of-cheap-badges-of-status}{}
\begin{center}\includegraphics[width=0.9\linewidth]{../Simulations/nIntGroupNormQual_/evolDyn7_nIntGroup2000} \end{center}
\end{frame}

\begin{frame}{Why would you want to be recognized?}
\protect\hypertarget{why-would-you-want-to-be-recognized}{}
\vspace{-1cm}

\begin{block}{The peaceful, the aggressive and the clever}
\protect\hypertarget{the-peaceful-the-aggressive-and-the-clever}{}
\begin{itemize}
\tightlist
\item
  Learning initial conditions
\end{itemize}

\includegraphics[width=0.45\linewidth]{../Simulations/alphaAct_/alphaAct0.4learnDyn}
\includegraphics[width=0.45\linewidth]{ISBE2022_files/figure-beamer/unnamed-chunk-24-2}
\end{block}
\end{frame}

\begin{frame}{Why would you want to be recognized?}
\protect\hypertarget{why-would-you-want-to-be-recognized-1}{}
\begin{block}{The peaceful, the aggressive and the clever}
\protect\hypertarget{the-peaceful-the-aggressive-and-the-clever-1}{}
\includegraphics[width=0.9\linewidth]{../Simulations/initAct_/indVarScatter_1}
\end{block}
\end{frame}

\begin{frame}{Why would you want to be recognized?}
\protect\hypertarget{why-would-you-want-to-be-recognized-2}{}
\begin{block}{The peaceful, the aggressive and the clever}
\protect\hypertarget{the-peaceful-the-aggressive-and-the-clever-2}{}
\includegraphics[width=0.9\linewidth]{../Simulations/initAct_/indVarScatter_}
\end{block}
\end{frame}

\begin{frame}{Take-home messages}
\protect\hypertarget{take-home-messages}{}
\begin{block}{Costly signals}
\protect\hypertarget{costly-signals}{}
\begin{itemize}
\tightlist
\item
  Learning can mediate costly signals
\end{itemize}
\end{block}

\begin{block}{Cheap signals}
\protect\hypertarget{cheap-signals}{}
\begin{itemize}
\tightlist
\item
  Learning mediates the evolution of polymorphisms with ``honest'' and
  ``dishonest'' individuals
\item
  Cheap signals depend on the innate level of aggression (before
  learning) of individuals
\end{itemize}
\end{block}
\end{frame}

\begin{frame}{Questions?}
\protect\hypertarget{questions}{}
\begin{center}\includegraphics[height=0.8\textheight]{../Images/BizarroParrot} \end{center}
\end{frame}

\begin{frame}{References}
\protect\hypertarget{references}{}
\hypertarget{refs}{}
\begin{CSLReferences}{1}{0}
\leavevmode\vadjust pre{\hypertarget{ref-grafen_Biological_1990}{}}%
Grafen, Alan. 1990. {``Biological Signals as Handicaps.''} \emph{Journal
of Theoretical Biology} 144 (4): 517--46.
\url{https://doi.org/10.1016/S0022-5193(05)80088-8}.

\leavevmode\vadjust pre{\hypertarget{ref-zahavi_Mate_1975}{}}%
Zahavi, Amotz. 1975. {``Mate Selection\textemdash -{A} Selection for a
Handicap.''} \emph{Journal of Theoretical Biology} 53 (1): 205--14.
\url{https://doi.org/10.1016/0022-5193(75)90111-3}.

\end{CSLReferences}
\end{frame}

\end{document}
